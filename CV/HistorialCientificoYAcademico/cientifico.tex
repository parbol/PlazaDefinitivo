\section{Introducción y notas preliminares}

Mi carrera investigadora se enmarca en el contexto de la Física Experimental de Altas Energías y más concretamente, se encuentra ligada al experimento Compact Muon Solenoid (CMS) del Large Hadron Collider (LHC) instalado en el CERN. La política interna de estas grandes colaboraciones establece que la autoría de las publicaciones, más de 1000 en el caso de CMS, pertenece a todos los miembros de la colaboración, apareciendo en orden alfabético en los artículos. Por este motivo soy autor formal de todas esas publicaciones y poseo un h-index de 162. En este curriculum se aportan como méritos un total de 60 artículos en revistas del primer tercil (primer cuartil en su mayoría) y contribuciones a 2 libros, correspondientes a aquellos trabajos en los que he tenido una contribución directa y significativa. 

\section{Líneas de investigación}

Mi trayectoria puede dividirse cronológicamente en tres grandes bloques: el periodo pre-doctoral (2005-2010) en el Instituto de Física de Cantabria (IFCA), el periodo post-doctoral (2010-2017) en el Instituto Tecnológico Federal Suizo en Zürich (ETHZ), y el periodo de Ramón y Cajal en el IFCA (2017-presente). Durante este tiempo he desempeñado posiciones de liderazgo en el \emph{commissioning} del sistema de muones de CMS; análisis de búsqueda de Supersimetría (SUSY) y otros modelos exóticos; diseño y construcción del MIPs Timing Detector (MTD); y desarrollo de la técnica de Tomografía Muónica. A lo largo de estos años he participado en un total de 16 proyectos de investigación y he tenido contribuciones en 45 congresos nacionales e internacionales.  


\subsection{\emph{Commisioning} del sistema de muones de CMS (2005-2010)}

Comencé mi doctorado en el IFCA con una ayuda para la realización de tesis doctorales otorgada por el CSIC. En este periodo tuve un papel de liderazgo en las actividades de \emph{commissioning} del sistema de muones de CMS, con contribuciones en la calibración de las cámaras de tubos de deriva, el desarrollo de algoritmos de reconstrucción de muones y el desarrollo de algoritmos de alineamiento para el sistema de muones. Las primeras geometrías alineadas usadas por el detector CMS en el comienzo del LHC fueron producidas por mí. La Colaboración CMS me otorgó el premio \emph{CMS Outstanding achievement award} y la Universidad de Cantabria (UC) el premio Extraordinario de doctorado por estos trabajos.


\subsection{Búsquedas de Supersimetría con el detector CMS (2010-presente)}

En el año 2010 me incorporé a la prestigiosa ETH Zürich y comencé a trabajar en búsquedas de Supersimetría (SUSY). Durante este tiempo he liderado a un grupo de postdocs y estudiantes de varias instituciones en búsquedas de SUSY con leptones del mismo sabor y signo opuesto. También he contribuído a otras búsquedas de SUSY, como las búsquedas con leptones del mismo signo, o búsquedas hadrónicas utilizando la variable MT2. Durante este periodo fui elegido en dos ocasiones para representar a CMS en la conferencia ICHEP (2012 y 2014) y también se me fue concedido el privilegio de presentar los primeros resultados públicos de la colaboración relacionados con búsquedas de SUSY a una energía de 13 TeV, en el prestigioso \emph{LHC Cern Seminar}. También fui elegido co-coordinador del grupo de \emph{Trigger, Montecarlo e Interpretaciones} del grupo de SUSY de CMS, estando a cargo del diseño e implementación de las estrategias de trigger y simulación de todo el grupo. Finalmente, la colaboración me eligió también como co-coordinador del grupo de búsquedas de tércera generación, contribuyendo activamente en varios de sus análisis.  

\subsection{Búsquedas \emph{exóticas} con el detector CMS (2017-presente)}

En 2017 me incorporé al IFCA como investigador Ramón y Cajal y comencé a trabajar en búsquedas de Materia Oscura. Desde entonces lideré las búsquedas de Materia Oscura en asociación con quark(s) top, en las que co-dirigido una tesis doctoral y estoy co-dirigiendo otra actualmente. Durante este periodo fui seleccionado para dar una charla plenaria con las búsquedas de Materia Oscura de CMS en la conferencia \emph{LHC Split Days}, entre otras. Recientemente, he comenzado una búsqueda de partículas de larga vida media en estados finales con leptones, en la que me encuentro co-dirigiendo otra tesis doctoral.


\subsection{El detector MTD de CMS (2017- presente)}

En 2018 comencé a trabajar en el proyecto para instalar un nuevo detector de tiempo, el MTD, en CMS. En el año 2019 fui elegido como representante español en el \emph{Institutional Board} y en el \emph{Financial Board} de dicho detector. También fui elegido como coordinador de los análisis de física del detector de cara a la elaboración del \emph{Technical Design Report} (TDR). En este contexto, tuve varias contribuciones en análisis de búsqueda de partículas de larga vida media y también en el análisis de producción de pares de Higgs, usando siempre el MTD para mejorar la \emph{sensitividad} de los análisis. También fui el editor del capítulo de \emph{Performance, Reconstruction and Physics} de dicho TDR. A finales del año 2019 fui elegido como co-coordinador L2 del gruop \emph{Data Performance Group} (DPG) del MTD, estando a cargo de todo el software, simulación, reconstrucción y evaluación del desempeño del detector. En el año 2020 conseguí incluir al MTD en el software de \emph{tracking} de CMS por primera vez. También en ese mismo año comencé las actividades para el ensamblado de módulos del MTD en el IFCA. En el año 2021 fui seleccionado para continuar como co-coordinador del MTD durante dos años más.  

\subsection{Muografía (2015-presente)}

En el año 2015, cofundé la compañía Muon Systems con el objeto de aplicar la muografía a problemas de la industria. Durante los años 2015 a 2017 tuve varias contribuciones como consultor tanto en el desarrollo de algoritmos de reconstrucción como en la fabricación de detectores de muones. Desde el año 2017, he sido pionero a nivel mundial en la aplicación de esta técnica, habiendo sido invitado a importantes foros como la \emph{Royal Society} o la Agencia Internacional de la Energía Atómica (AIEA), como representante español. En este contexto soy editor del documento técnico que la AIEA está elaborando para ofrecer como recomendación a sus países miembros en lo que se refiere a la muografía. Durante este periodo he tenido varias contribuciones a \emph{workshops} y conferencias y he sido invitado a varios seminarios. También he sido Investigador Principal de un proyecto en el contexto de un convenio entre Muon Systems y la Universidad de Cantabria. Actualmente soy Investigador Principal de un proyecto para producir simulaciones ultrarápidas en el contexto de la muografía utilizando redes generativas adversarias. Finalmente, soy también miembro de la colaboración internacional MODE, que busca la aplicación de técnicas de \emph{differential programming} para la optimización del diseño de detectores de partículas y concretamente en el contexto de la muografía. 
