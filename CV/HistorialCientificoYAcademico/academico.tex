\section{Introducción}

Mi experiencia docente comenzó en el año 2010 cuando, después de doctorarme, ingresé como profesor asistente e investigador postdoctoral en el Instituto Tecnológico Federal Suizo en Zúrich (ETHZ) y ha continuado desde el año 2017, en el que me incorporé a la Universidad de Cantabria (UC) como Investigador Ramón y Cajal. A día de hoy he impartido un total de 900 horas de docencia: 456 en la ETHZ y 444 en la
Universidad de Cantabria, cubriendo un total de 12 cursos académicos y 10 asignaturas diferentes. También he dirigido una tesis doctoral, 5 Trabajos de Fin de Grado y 7 Trabajos de Fin de Máster. 

Mi actividad docente ha estado vinculada al grado de Física o a cursos de postgrado (Máster) en Física de Partículas y del Cosmos o de Ciencia de Datos. En este sentido ha existido y existe una vinculación entre mi actividad investigadora como físico de partículas experimental experto en análisis de datos y mi actividad docente. También me gustaría resaltar que he tenido experiencia docente tanto en los primeros cursos de grado como en los últimos, dando asignaturas de laboratorio y también de teoría. Toda mi actividad docente ha estado contenida en programas de evaluación de la calidad docente.

\section{Docencia en la ETH Zürich}

Toda la docencia impartida en la ETHZ tuvo como lengua oficial el Inglés. El total de horas de clase asciende a 456. A las clases presenciales hay que añadir la dirección de un Trabajo de Fin de Máster relacionado con una de las búsquedas de SUSY en las que trabaja en ese momento, tal y como se detalla en el curriculum anexo a este historial. 

\subsection{Docencia de grado en asignaturas de teoría}  

\subsubsection{Physics I (Año 2011) - 24 horas y Physics II (Año 2012 y 2013) - 48 horas}

Estas asignaturas se corresponden con los cursos de introducción a la Física para estudiantes de primer año del grado de física. El reto principal de la asignatura consistía en el número de estudiantes que ascendía típicamente a unos 400 alumnos repartidos en clases de hasta 50 o 60, estando yo a cargo de una de ellas. La coordinación entre los diferentes miembros del equipo docente fue fundamental y destacable, manteniendo reuniones semanales para hacer que el progreso de la asignatura fuera uniforme. Cabe destacar que en la asignatura de Physics II obtuve una evaluación de mi actividad docente correspondiente con 4.6 / 5.0 y 4.8 / 5.0 (ver certificado de docencia de la ETH).

\subsubsection{Introduction to Nuclear and Particle Physics (Años 2011, 2014, 2015 y 2016) - 128 horas}

Esta asignatura se corresponde con el último curso (en la ETHZ) del grado de Física y cubre los principios básicos de la física nuclear y de partículas, especialmente desde una perspectiva experimental y fenomenológica. La conexión con mi ámbito de investigación es evidente lo cual siempre permitió la incorporación en las clases de material adicional directamente relacionado con los últimos avances en Física de Partículas. El número de alumnos ascendía a unos 30-40 por clase, dependiendo del año. 

\subsection{Docencia en grado en asignaturas de laboratorio}

\subsubsection{Physics Lab I (Año 2012, 2015 y 2016) - 192 horas}

Esta asignatura, obligatoria para los estudiantes de primer año de Física (en la ETHZ), tiene como reto también el gran número de alumnos (en torno a 400) lo cuál hacía necesaria la perfecta coordinación del equipo docente a través de reuniones bi-semanales. La asignatura introduce a los alumnos de primer año en las técnicas de laboratorio y sobre todo el tratamiento y análisis estadístico de los datos. De nuevo, esta disciplina, estuvo siempre en consonancia con mi actividad investigadora, usando continuamente ejemplos reales de mi experiencia para ilustrar la importancia del correcto análisis de los datos y su interpretación.

\subsubsection{Advanced Physics Lab I (Año 2014) - 64 horas}

Este laboratorio es obligatorio para los estudiantes de tercer año de Física (en la ETHZ). El laboratorio puede verse como una extensión del curso Physics Lab I, con experimentos más complejos y sofisticados y análisis de datos más elaborados.


\section{Docencia en la Universidad de Cantabria}

La docencia en la Universidad de Cantabria fue impartida en lengua española e inglesa. La valoración total obtenida por parte de la Universidad de Cantabria en cuanto a mi actividad docente es de 4.7/5.0 calificada como de MUY FAVORABLE. A las clases presenciales hay que añadir también la dirección de 5 Trabajos de Fin de Grado y 6 Trabajos de Fin de Máster con temáticas relacionadas siempre con mi trabajo investigador: desarrollo de ideas nuevas en el contexto de las búsquedas de SUSY o Materia Oscura, nueva algoritmia en el contexto de la Muografía, o instrumentación para el detector MTD. El curriculum anexo contiene una lista detallada de estos trabajos.

\subsection{Docencia en grado en asignaturas de teoría}

\subsubsection{Física de Partículas elementales (Años 2017, 2018, 2019, 2020, 2021) - 174 horas }

Esta asignatura, de la que soy responsable, pertenece al cuarto curso del grado en Física y está directamente relacionada con mi actividad investigadora al tener como contenidos la formulación matemática del Modelo Estándar de Física de Partículas.

\subsubsection{Mecánica Cuántica (Años 2020, 2021) - 50 horas}

Esta asignatura provee a los alumnos con los fundamentos teóricos y el formalismo matemático de la Mecánica Cuántica. La asignatura se encuentra muy relacionada con la de Física de Partículas elementales, así que resulta importante mantener una buena sincronización en términos de explicaciones y notaciones. 

\subsection{Docencia en grado en asignaturas de laboratorio}

\subsubsection{Advanced Experimental Techniques (Años 2017, 2018) - 66.8 horas}

Esta asignatura, impartida en inglés, se cursa en cuarto año de grado y es muy similar a la asignatura de Laboratorio Avanzado impartida en la ETHZ. Cabe destacar que en esta asignatura yo fui responsable de un experimento relacionado con la detección de muones cósmicos de la que soy experto y que guarda relación con mis actividades de transferencia tecnológica además de investigadora.

\subsection{Docencia en postgrado en asignaturas de teoría}

\subsubsection{Máster de Ciencia de Datos (UC-UIMP) - Estadística para la Ciencia de
Datos (Años 2017, 2018 y 2019, 2020 y 2021) - 89 horas}

Esta asignatura troncal del Máster de Ciencia de Datos da una introducción a conceptos estadísticos para el análisis de datos. Gracias a mi campo de especialización en
el análisis de datos del LHC la asignatura proporciona ejemplos modernos aplicados a dicho campo entre otros. Cabe destacar también que participé en esta asignatura desde el comienzo del Máster habiendo creado todo el material docente.

\subsubsection{Máster de Física de Partículas y del Cosmos (UC-UIMP) - Modelo Estándar de Física de Partículas - 20 horas}

Esta asignatura, impartida en inglés, es una extensión de la asignatura “Física de Partículas Fundamentales”, incidiendo en aspectos concretos de la formulación matemática del Modelo Estándar y perteneciendo a mi ámbito de investigación.

\section{Otros méritos docentes}

\subsection{Profesor del \emph{CMS Data Analysis School} en Pisa}

En el año 2012, CMS me seleccionó como profesor de la \emph{CMS Data Analysis School} celebrada en Pisa. Esta escuela tiene como objetivo introducir a doctorandos y jóvenes postdocs en las técnicas de análisis del detector CMS. Mi contribución consistió en 32 horas de clase en las que introduje a un grupo de 10 estudiantes los fundamentos de las técnicas de búsqueda de Supersimetría. Con este objetivo, no sólo impartí las clases, sino que preparé un ejercicio completo e interactivo en el que los estudiantes podían hacer por sí mismos todos los pasos del análisis, de manera simplificada, cubriendo todos los aspectos del mismo.

\subsection{Actividades de divulgación}

En los últimos años he desarrollado también una intensa actividad divulgadora. He participado en eventos de alto impacto como \emph{Pint of Science} o \emph{Aquae Talent Hub}, en Cafés Científicos, tanto en Cantabria como fuera de la comunidad, y también soy miembro activo del programa de \emph{Expandiendo la ciencia}, con más de siete charlas en colegios e institutos de la región, o en la Noche de los Investigadores organizada por la Universidad de Cantabria.



