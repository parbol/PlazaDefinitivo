%%%%%%%%%%%%%%%%%%%%%%%%%%%%%%%%%%%%%%%%%
% Short Sectioned Assignment
% LaTeX Template
% Version 1.0 (5/5/12)
%
% This template has been downloaded from:
% http://www.LaTeXTemplates.com
%
% Original author:
% Frits Wenneker (http://www.howtotex.com)
%
% License:
% CC BY-NC-SA 3.0 (http://creativecommons.org/licenses/by-nc-sa/3.0/)
%
%%%%%%%%%%%%%%%%%%%%%%%%%%%%%%%%%%%%%%%%%

%----------------------------------------------------------------------------------------
%	PACKAGES AND OTHER DOCUMENT CONFIGURATIONS
%----------------------------------------------------------------------------------------

\documentclass[paper=a4, fontsize=11pt]{scrartcl} % A4 paper and 11pt font size

\usepackage[T1]{fontenc} % Use 8-bit encoding that has 256 glyphs
\usepackage{fourier} % Use the Adobe Utopia font for the document - comment this line to return to the LaTeX default
\usepackage[spanish]{babel} % English language/hyphenation
\selectlanguage{spanish}
\usepackage[utf8]{inputenc}
\usepackage{amsmath,amsfonts,amsthm} % Math packages
\usepackage{graphicx}

\usepackage{sectsty} % Allows customizing section commands
\allsectionsfont{\centering \normalfont\scshape} % Make all sections centered, the default font and small caps

\usepackage{fancyhdr} % Custom headers and footers
\pagestyle{fancyplain} % Makes all pages in the document conform to the custom headers and footers
\date{}
\fancyhead{} % No page header - if you want one, create it in the same way as the footers below
\fancyfoot[L]{} % Empty left footer
\fancyfoot[C]{} % Empty center footer
\fancyfoot[R]{\thepage} % Page numbering for right footer
\renewcommand{\headrulewidth}{0pt} % Remove header underlines
\renewcommand{\footrulewidth}{0pt} % Remove footer underlines
\setlength{\headheight}{5.6pt} % Customize the height of the header

\numberwithin{equation}{section} % Number equations within sections (i.e. 1.1, 1.2, 2.1, 2.2 instead of 1, 2, 3, 4)
\numberwithin{figure}{section} % Number figures within sections (i.e. 1.1, 1.2, 2.1, 2.2 instead of 1, 2, 3, 4)
\numberwithin{table}{section} % Number tables within sections (i.e. 1.1, 1.2, 2.1, 2.2 instead of 1, 2, 3, 4)

\setlength\parindent{0pt} % Removes all indentation from paragraphs - comment this line for an assignment with lots of text

%----------------------------------------------------------------------------------------
%	TITLE SECTION
%----------------------------------------------------------------------------------------

\newcommand{\horrule}[1]{\rule{\linewidth}{#1}} % Create horizontal rule command with 1 argument of height

\title{	
\normalfont \normalsize 
\textsc{UNIVERSIDAD DE CANTABRIA, DEPARTAMENTO DE FÍSICA MODERNA} \\ [20pt] % Your university, school and/or department name(s)
\horrule{0.5pt} \\[0.4cm] % Thin top horizontal rule
\huge Física de Partículas Elementales (G71) \\ % The assignment title
\normalsize 4 Curso - Grado de Física - Doble Grado Física Matemáticas - Ejercicios Tema 5
\horrule{2pt} \\[0.5cm] % Thick bottom horizontal rule
}

\begin{document}

\maketitle % Print the title

\vspace{-2.5cm}

%----------------------------------------------------------------------------------------
%	PROBLEM 1
%----------------------------------------------------------------------------------------
\textbf{Cuestión 1.} Usando las amplitudes para estados de helicidad calcular la sección eficaz diferencial para el proceso $e+\mu\rightarrow e+\mu$, siguiendo
los siguientes pasos:

\begin{enumerate}
\item Usando las reglas de Feynman para QED mostrar que el elemento de matriz de más bajo orden es:
\begin{equation*}
M_{fi} = -\frac{e^2}{(p_1-p_3)^2}g_{\mu\nu}[\bar{u}(p_3)\gamma^\mu u(p_1)][\bar{u}(p_4)\gamma^\nu u(p_2)]
\end{equation*}

\item Trabajando en el sistema de referencia del centro de masas y escribiendo los cuadrimomentos del electrón inicial y final como $p_1^\mu = (E_1, 0, 0, p)$ y $p_3^\mu = (E_1, p sin(\theta), 0, p cos(\theta))$, demuestra que las corrientes asociadas al electrón para las cuatro helicidades pueden escribirse como:

\begin{eqnarray*}
\bar{u_{\downarrow}}(p_3)\gamma^\mu u_{\downarrow}(p_1) &=& 2 (E_1 c, ps, -ips, pc) \\
\bar{u_{\uparrow}}(p_3)\gamma^\mu u_{\downarrow}(p_1) &=& 2 (ms, 0, 0, 0) \\
\bar{u_{\uparrow}}(p_3)\gamma^\mu u_{\uparrow}(p_1) &=& 2 (E_1 c, ps, ips, pc) \\
\bar{u_{\downarrow}}(p_3)\gamma^\mu u_{\uparrow}(p_1) &=& -2 (ms, 0, 0, 0)
\end{eqnarray*}

siendo $s = sin(\theta / 2)$ y $c = cos(\theta/2)$.

\item Explicar por qué el efecto de aplicar el operador de paridad $P=\gamma^0$ es:

\begin{equation*}
P{u_{\uparrow}}(p, \theta, \phi) = u_{\downarrow}(p, \pi - \theta, \pi + \phi) 
\end{equation*}

y calcula haciendo uso de ello las corrientes asociadas a los muones de las distintas combinaciones de helicidad.

\item Para el caso relativista $E>>M$, muestra que el elemento de matriz al cuadrado para el caso en el que tanto el electrón como el muon incidentes son left-handed, está dado por: 

\begin{equation*}
|M_{LL}|^2 = \frac{4e^2s^2}{(p_1-p_3)^4}
\end{equation*}

donde $s = (p_1 + p_ 2)^2$. Hallas las expresiones correspondientes para $M_{RR}$, $M_{RL}$ y $M_{LR}$.

\item En el límite relativista demuestra que la sección eficaz diferencial para este proceso en el caso no polarizado en el sistema centro de masas es:

\begin{equation*}
\frac{d\sigma}{d\Omega} = \frac{2\alpha^2}{s} \frac{1 + 1/4(1 + cos(\theta))^2}{(1-cos(\theta))^2}
\end{equation*}

\end{enumerate}

%----------------------------------------------------------------------------------------
%	PROBLEM 2
%----------------------------------------------------------------------------------------
\textbf{Cuestión 2.} Demuestra que los operadores de proyección quiral cumplen: $P_R + P_L = I$, $P_RP_R=P_R$, $P_LP_L=P_L$ y $P_RP_L=0$. 
\\
\\
%----------------------------------------------------------------------------------------
%	PROBLEM 3
%----------------------------------------------------------------------------------------
\textbf{Cuestión 3.} Demuestra que:

\begin{equation*}
\Lambda^+ = \frac{m+\gamma^\mu p_\mu}{2m}\; \Lambda^- = \frac{m-\gamma^\mu p_\mu}{2m} 
\end{equation*}

son también operadores de proyección y demuestra que proyectan sobre los estados de partícula y antipartícula respectivamente.

\begin{equation*}
\Lambda^+ u= u, \Lambda^- v= v, \Lambda^+ v = \Lambda^- u = 0 
\end{equation*}


\end{document}
