%%%%%%%%%%%%%%%%%%%%%%%%%%%%%%%%%%%%%%%%%
% Short Sectioned Assignment
% LaTeX Template
% Version 1.0 (5/5/12)
%
% This template has been downloaded from:
% http://www.LaTeXTemplates.com
%
% Original author:
% Frits Wenneker (http://www.howtotex.com)
%
% License:
% CC BY-NC-SA 3.0 (http://creativecommons.org/licenses/by-nc-sa/3.0/)
%
%%%%%%%%%%%%%%%%%%%%%%%%%%%%%%%%%%%%%%%%%

%----------------------------------------------------------------------------------------
%	PACKAGES AND OTHER DOCUMENT CONFIGURATIONS
%----------------------------------------------------------------------------------------

\documentclass[paper=a4, fontsize=11pt]{scrartcl} % A4 paper and 11pt font size

\usepackage[T1]{fontenc} % Use 8-bit encoding that has 256 glyphs
\usepackage{fourier} % Use the Adobe Utopia font for the document - comment this line to return to the LaTeX default
\usepackage[spanish]{babel} % English language/hyphenation
\selectlanguage{spanish}
\usepackage[utf8]{inputenc}
\usepackage{amsmath,amsfonts,amsthm} % Math packages
\usepackage{graphicx}

\usepackage{sectsty} % Allows customizing section commands
\allsectionsfont{\centering \normalfont\scshape} % Make all sections centered, the default font and small caps

\usepackage{fancyhdr} % Custom headers and footers
\pagestyle{fancyplain} % Makes all pages in the document conform to the custom headers and footers
\date{}
\fancyhead{} % No page header - if you want one, create it in the same way as the footers below
\fancyfoot[L]{} % Empty left footer
\fancyfoot[C]{} % Empty center footer
\fancyfoot[R]{\thepage} % Page numbering for right footer
\renewcommand{\headrulewidth}{0pt} % Remove header underlines
\renewcommand{\footrulewidth}{0pt} % Remove footer underlines
\setlength{\headheight}{5.6pt} % Customize the height of the header

\numberwithin{equation}{section} % Number equations within sections (i.e. 1.1, 1.2, 2.1, 2.2 instead of 1, 2, 3, 4)
\numberwithin{figure}{section} % Number figures within sections (i.e. 1.1, 1.2, 2.1, 2.2 instead of 1, 2, 3, 4)
\numberwithin{table}{section} % Number tables within sections (i.e. 1.1, 1.2, 2.1, 2.2 instead of 1, 2, 3, 4)

\setlength\parindent{0pt} % Removes all indentation from paragraphs - comment this line for an assignment with lots of text

%----------------------------------------------------------------------------------------
%	TITLE SECTION
%----------------------------------------------------------------------------------------

\newcommand{\horrule}[1]{\rule{\linewidth}{#1}} % Create horizontal rule command with 1 argument of height

\title{	
\normalfont \normalsize 
\textsc{UNIVERSIDAD DE CANTABRIA, DEPARTAMENTO DE FÍSICA MODERNA} \\ [20pt] % Your university, school and/or department name(s)
\horrule{0.5pt} \\[0.4cm] % Thin top horizontal rule
\huge Física de Partículas Elementales (G71) \\ % The assignment title
\normalsize 4 Curso - Grado de Física - Doble Grado Física Matemáticas - Ejercicios Tema 2
\horrule{2pt} \\[0.5cm] % Thick bottom horizontal rule
}

\begin{document}

\maketitle % Print the title

\vspace{-2.5cm}

%----------------------------------------------------------------------------------------
%	PROBLEM 1
%----------------------------------------------------------------------------------------
\textbf{Cuestión 1.} El índice de refracción en el agua es de n = 1.333. Calcula la energía mínima en KeV de un electrón para producir luz Cherenkov.
\\
\\
%----------------------------------------------------------------------------------------
%       PROBLEM 2
%----------------------------------------------------------------------------------------
\textbf{Cuestión 2.} Estima la longitud de radiación específica para el hierro (A = 56, Z = 26 and $\rho$=7.8 g/cm$^3$. 
\\
\\
%----------------------------------------------------------------------------------------
%       PROBLEM 3
%----------------------------------------------------------------------------------------
\textbf{Cuestión 3.} Explica las principales características de la ecuación de Bethe-Bloch (BB). ¿Para qué valores de $\beta\gamma$ la ecuación tiene un mínimo?¿Cómo se llaman esas partículas?
¿Cuál es la dE/dx típica para ese mínimo?.
\\
\\
%----------------------------------------------------------------------------------------
%       PROBLEM 4
%----------------------------------------------------------------------------------------
\textbf{Cuestión 4.} Cálcula la pérdida de energía de una partícula mínimamente ionizante (MIP) en silicio y en árgon. Note: $\rho$(Si) = 2.329 g/cm$^3$ y $\rho$(Ar) = 0.001662 g/cm$^3$.
\\
\\
%----------------------------------------------------------------------------------------
%       PROBLEM 5
%----------------------------------------------------------------------------------------
\textbf{Cuestión 5.} Típicamente un sensor de silicio en un detector de píxeles híbrido tiene 300 micras de espesor, mientras que una cámara multihilo tiene típicamente una anchura de 
30~cm. ¿Cuánta energía se deposita en cada detector, si una partícula mínimamente ionizante atraviesa los detectores perpendicularmente?
\\
\\
%----------------------------------------------------------------------------------------
%       PROBLEM 6
%----------------------------------------------------------------------------------------
\textbf{Cuestión 6.} Un detector cilíndrico es instalado en torno a un punto de interacción que queda ubicado en el centro del cilindro. Un campo magnético constante y paralelo al 
eje del cilindro es aplicado con magnitud B. Si el radio del detector es R = 3.5~m y B = 1 T. ¿Cuál es el mínimo momento transverso de una partícula para ser capaz de abandonar el detector?

\end{document}
